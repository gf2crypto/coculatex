%%= theme: dmarticle.ru
%%= project-name: subcodes_corank_1
%%= title: Hadamard products classification of subcodes of Reed--Muller codes codimension 1
%%= authors:
%%=    -
%%=       name: Ivan Chizhov
%%=       institute: Lomonosov Moscow State University
%%=       email: ivan@email.edu.ru
%%=    -
%%=       name: Mikhail Borodin
%%=       institute: JSC “InfoTeCS”
%%=       email: misha@inft.domain.ru
%%= keywords: McEliece public key cryptosystem, code-based cryptosystems, Reed--Muller codes, subcodes of codimension 1, square code attack, component-wise product, Hadamard product, Shur product, square of code, Hadamard product classification, cryptanalysis
%%= abstract: |
%%=     Subcodes of Reed--Muller codes are considered.
%%=     The Hadamard products classification of such subcodes is obtained.
%%      By using this classification, authors proved that in most cases the task of the secret key recovering of a cryptosystem which is based on such subcodes is equivalent to the task of the secret key recovering of the same cryptosystem but which is based on the Reed--Muller codes.
%%      And for the McEliece public key cryptosystem based on Reed--Muller codes exists effective structural attacks.
%%= ru:
%%=     title: Классификация произведений Адамара подкодов  коразмерности 1 кодов Рида--Маллера
%%=     authors:
%%=     -
%%=         name: И.В.~Чижов
%%=         institute: МГУ имени М.В.~Ломоносова
%%=         email: ichizhov@cs.msu.ru
%%=     -
%%=         name: М.А.~Бородин
%%=         institute: ОАО <<ИнфоТеКС>>
%%=         email: Mikhail.Borodin@infotecs.ru
%%=     udk: УТОЧНИТЬ
%%=     keywords: криптосистема Мак-Элиса, кодовые криптосистемы, коды Рида--Маллера, криптоанализ, атака
%%=     abstract: |
%%=         В работе рассматривались подкоды коразмерности 1 кодов Рида--Маллера.
%%=         Получена классификация произведений Адамара таких подкодов.
%%=         С помощью этой классификации авторам удалось установить, что в большинстве случаев задача восстановления секретного ключа кодовой криптосистемы, построенной на основе таких подкодов эквивалентна задаче восстановления секретного ключа этой же криптосистемы, но построенной на самих кода Рида--Маллера, для которой имеются достаточно эффективные алгоритмы взлома.
%%= tex_preambule: |
%%=     \DeclareMathOperator{\wt}{wt}
%%=     \DeclareMathOperator{\supp}{supp}

\section{Введение}

Криптосистема Мак-Элиса была предложена в 1978 году Р.~Дж.~Мак-Элисом~\cite{mceliece1978public}.
Стойкость этой криптосистемы основана на предположении сложности декодирования кода общего положения, т.е. кода, не обладающего видимой алгебраической и комбинаторной структурой.
Оригинальная криптосистема Мак-Элиса строится на двоичных кодах Гоппы.
Для ускорения процедур шифрования и расшифрования, а также увеличения скорости передачи криптосистемы (уменьшения избыточности шифр-текста), В.~М.~Сидельников предложил использовать коды Рида--Маллера~\cite{sidelnikov1994open}.
Однако в 2007 году Л.~Миндера и А.~Шокроллахи предложили достаточно эффективную атаку на такую криптосистему~\cite{minder2007cryptanalysis}.
Кроме того, в 2014 году М.~А.~Бородин и И.~В.~Чижов~\cite{borodin2014effective} усовершенствовали атаку Миндера--Шокроллахи и построили полиномиальную атаку в случае использования кода Рида--Маллера \(RM(r,m)\)  с параметрами, удовлетворяющими условию НОД\((r,m-1)=1\).

Часто для повышения стойкости кодовых криптосистем пытаются перейти к подкодам известных кодов, нарушив тем самым структуры, используемые для построения атак.
Например, в работе~\cite{berger2005how} была предложена криптосистема Берже--Луадро (Berger--Loidreau), построенная на подкодах кода Рида--Соломона, и проведён криптоанализ этой системы, из которого следует, что для указанной криптосистемы атака Сидельникова-Шестакова~\cite{sidelnikov1992about} является неэффективной.
Тем не менее, в работах~\cite{wieschebrink2006attack} и~\cite{wieschebrink2010cryptanalysis} показано, как построить в некоторых случаях эффективные атаки на криптосистему Берже--Луадро.

Основным инструментом атак Чижова--Бородина и Вишебринка является аппарат произведений Адамара двух кодов.
Кроме того, этот аппарат эффективно используется для построения атак на криптосистему Мак-Элиса, основанную на достаточно большом типе кодов.
Так в работе~\cite{couvreur2015cryptanalysis} предлагается эффективная атака на криптосистему, построенную на подкодах алгебро-геометрических кодов.
В~\cite{couvreur2014distinguisher} авторы строят на основе произведений Адамара кодов различитель между случайными кодами и подкодами кодов Рида--Соломона специального вида.
В работе~\cite{otmani2015square} установлено, что криптосистема Мак-Элиса, построенная на модификации кода Рида--Маллера с помощью добавления случайных координат к кодовым словам, эквивалентна по стойкости оригинальной криптосистеме на обычных кодах Рида--Маллера.
В~\cite{couvreur2015polynomial} предлагается полиномиальная атака на модификацию криптосистемы Мак-Элиса, использующую коды Рида--Соломона, а в работе~\cite{couvreur2017polynomial} строится атака на оригинальную криптосистему Мак-Элиса, основанную на кодах Гоппы над квадратичными расширениями конечных полей.

В работе рассматриваются подпространства коразмерности 1 кодов Рида--Маллера.
Главным результатом работы является построение полной классификации произведений Адамара таких подкодов относительно группы автоморфизмов кодов Рида--Маллера.
Кроме того, полученная классификация применяется к криптоанализу двух аналогов криптосистемы Берже--Луадро, но построенных не на покодах кода Рида--Соломона, а на подкодах коразмерности 1 кодов Рида--Маллера.
В итоге во многих случаях удаётся строго доказать, что эти криптосистемы не отличаются по стойкости от криптосистемы Мак-Элиса, построенной на основе кодов Рида--Маллера.

\section{Основные понятия}

Пусть \(V_n=\{0,1\}^n\)~---пространство всех двоичных векторов длины \(n\).
\emph{Двоичным блоковым кодом} или просто \emph{кодом} будем называть произвольное линейное векторное подпространство пространства \(V_n\).

В работе рассматриваются только коды Рида--Маллера и производные из них коды.

Рассмотрим булеву функцию
\[f(x_1,x_2,\ldots,x_m ):V_m \rightarrow \{0,1\},\]
известно~\cite{mcwilliams1979theory}, что её можно однозначно задавать вектором значений:
\[
	(f(0,0,\ldots,0),f(0,0,\ldots,1),\ldots,f(1,1,\ldots,1)).
\]
Другим представлением булевой функции является \emph{полином Жегалкина} или \emph{алгебраическая нормальная форма}.
Если  $x_{i_1},x_{i_2},\ldots,x_{i_d}$~--- различные символы переменных, то булеву функцию $x_{i_1}\cdot x_{i_2}\cdot \ldots \cdot x_{i_d}$ будем называть \emph{мономом}, а число $d$~--- его \emph{степенью}.
Константу $1$ будем считать мономом степени $0$.
Для мономов от $m$ переменных можно использовать <<степенную>> форму записи: $x^\alpha=x_1^{\alpha_1}x_2^{\alpha_2}\ldots x_m^{\alpha_m}$, здесь для любого $0\leqslant i\leqslant m$ $\alpha_i$ принимает значение либо $0$, либо $1$, а $x_i^{\alpha_i}$ равно $x_i$, если $\alpha_i=1$, и $1$ в противном случае.
Таким образом, каждому моному от $m$ переменных можно однозначно поставить в соответствие двоичный вектор $\alpha$ длины $m$, причём координата $\alpha_{i}$ этого вектора будет равна $1$ в том и только том случае, когда переменная $x_i$ входит в моном.
При этом очевидно, что степень монома будет равна весу Хэмминга вектора $\alpha$.

Умножение мономов \(x^{\alpha}\) и \(x^{\beta}\) также можно выразить через операцию \(\vee\) логического <<или>> над степенями.
Так
\begin{equation*}
	x^{\alpha}\cdot x^{\beta} = x^{\alpha \vee\beta},\text{ где } \supp(\alpha\vee\beta)=\supp(\alpha)\cup\supp(\beta).
\end{equation*}

Известно~\cite{mcwilliams1979theory}, что произвольная булева функция может быть представлена в виде следующей суммы по модулю 2
\begin{equation}
	\label{anf}
	f(x_{{1}},x_2,\ldots,x_m)=\bigoplus_{\alpha=(\alpha_{1},\alpha_2,\ldots,\alpha_m)\in V_m}a_{\alpha}x^{\alpha},
\end{equation}
где \(a_{\alpha}, \alpha\in V_m\)~--- константа из множества $\{0,1\}$.
В дальнейшем не будем делать различий между представлениями булевых функций в виде вектора значений и полинома Жегалкина.

\emph{Степенью} булевой функции, $\deg(f)$, называется максимальная степень мономов, входящих в её полином Жегалкина~(\ref{anf}).
Степень также равна такому наименьшему целому положительному числу \(d\), что в представлении~(\ref{anf}) все \(a_{\alpha}=0\) для \(\alpha\) веса Хэмминга строго большего \(d\), \(\wt(\alpha)> d\).

\emph{Кодом Рида--Маллера} \(RM(r,m)\) называется множество всех булевых функций от \(m\) переменных, у которых \(\deg(f)\leqslant r\).
Известно~\cite{mcwilliams1979theory}, что базисом кода, в частности, являются все мономы степени от 0 до \(r\) включительно от \(m\) переменных:
\begin{equation}
	\label{BasesRM}
	1, x_1,\ldots ,x_{m},x_1x_2,\ldots,x_{m-1}x_m,\ldots,x_{1}x_2\cdots x_r,\ldots,x_{m-r-1}\cdots x_m
\end{equation}
Длина кода Рида--Маллера равна~\cite{mcwilliams1979theory} \(n=2^m\), а размерность равна~\cite{mcwilliams1979theory} $k=\sum_{i=0}^{r}{{m}\choose{i}}$.

Определим носитель вектора $\alpha\in V_m$, как множество $\supp(\alpha)=\{i\in \{1,2,...,m\} | \alpha_i = 1\}$, состоящее из номеров ненулевых координат вектора $\alpha$.

Для двоичного набора \(\alpha=(\alpha_{1},\ldots,\alpha_{m})\) символом \(\|\alpha\|\) будем обозначать его представление в виде десятичного числа, т.е.
$\|\alpha\|=\alpha_{m}+2\alpha_{m-1}+\ldots+2^{m-1}\alpha_{1}.$

Введем отношение строгого порядка для векторов из $V_m$.
Для $\alpha,\beta \in V_m$ будем считать, что $\alpha < \beta$, если, либо $\wt(\alpha) < \wt(\beta)$, либо $\wt(\alpha) = \wt(\beta)$ и $\|\alpha\| < \|\beta\|$.
Тогда можно ввести отношение строгого порядка на множестве мономов: $x^{\alpha}<x^{\beta}$, если и только если $\alpha<\beta$.
Введённое отношение порядка обладает рядом важных в дальнейшем свойств.

\begin{proposition}
	\label{prop_order}
	Пусть $\alpha, \beta, \gamma, \delta \in V_m$.
	Тогда если $x^{\alpha} < x^{\gamma}$, $x^{\beta} \leqslant x^{\delta}$ и $\supp(\gamma) \cap \supp(\delta)=\emptyset$, то $x^{\alpha} \cdot x^{\beta} < x^{\gamma}\cdot x^{\delta}$.
\end{proposition}
\begin{proof}
	Во-первых, по построению порядка \(|\supp(\alpha)|\leqslant|\supp(\gamma)|\) и \(|\supp(\beta)|\leqslant |\supp(\delta)|\).
	Во-вторых, \(|\supp(\gamma)\cup \supp(\delta)|=|\supp(\gamma)|+|\supp(\delta)|\), т.к. множества \(\supp(\gamma)\) и \(\supp(\delta)\) по условию не пересекаются.
	Тогда
	\begin{equation}
		\begin{split}
			\label{eq_prop}
			|\supp(\alpha)\cup\supp(\beta)| &\leqslant |\supp(\alpha)| + |\supp(\beta)| \leqslant |\supp(\gamma)| + |\supp(\delta)|\\
			&= |\supp(\gamma)\cup\supp(\delta)|
		\end{split}
	\end{equation}
	Когда в (\ref{eq_prop}) хотя бы одно неравенство строгое справедливость утверждения прямо следует из построения порядка.

	Пусть теперь в (\ref{eq_prop}) все неравенства обращаются в равенства.
	Это в частности означает, что множества \(\supp(\alpha)\) и \(\supp(\beta))\) не пересекаются, а также что \(|\supp(\alpha)|=|\supp(\gamma)|\) и \(|\supp(\beta)|=|\supp(\delta)|\).
	Но тогда из построения порядка \(\|\alpha\| < \|\gamma\|\) и \(\|\beta\|\leqslant\|\ \delta\|\).

	По правилам сравнения двоичных чисел, у которых старшие разряды находятся слева, найдётся такой минимальный номер \(1\leqslant i\leqslant m\), что все координаты наборов \(\alpha\) и \(\gamma\) с меньшим номером равны, а координата с номером \(i\) набора \(\gamma\) равна 1, а набора \(\alpha\)~--- 0.
	По условию носители наборов \(\gamma\) и \(\delta\) не пересекаются, поэтому \(\delta_i = 0\).
	Если \(\beta=\delta\), то в этом случае \(\|\alpha\vee\beta\|<\|\gamma\vee\delta\|\), т.к.
	\((\alpha_1,\ldots,\alpha_{i-1})=(\gamma_1,\ldots,\gamma_{i-1})\), \((\beta_1,\ldots,\beta_{i-1})=(\delta_1,\ldots,\delta_{i-1})\) и \(\gamma_{i}=1\), а \(\alpha_{i}=\beta_{i}=\delta_{i}=0.\)

	При \(\|\beta\|<\|\delta\|\) как и для наборов \(\alpha\) и \(\gamma\) найдётся минимальный номер \(1\leqslant j\leqslant m\), что \((\beta_1,\ldots,\beta_{j-1})=(\delta_1,\ldots,\delta_{j-1})\), а \(\beta_j=0\) и \(\delta_j=1\).
	Заметим, что \(j\) не может быть равен номеру \(i\), т.к.
	\(\delta_i=0\).
	Когда \(i < j\) выполняется равенство  \((\beta_1,\ldots,\beta_{i}=0)=(\delta_1,\ldots,\delta_{i}=0)\), а значит и \(\|\alpha\vee\beta\|<\|\gamma\vee\delta\|\).
	В случае \(i > j\) при условии, что носители векторов \(\gamma\) и \(\delta\) не пересекаются, выполняется \((\beta_1,\ldots,\beta_{j-1})=(\delta_1,\ldots,\delta_{j-1})\), \((\alpha_1,\ldots,\alpha_{j-1})=(\gamma_1,\ldots,\gamma_{j-1})\),  и \(\delta_{j}=1\), а \(\alpha_{j}=\beta_{j}=\gamma_{j}=0,\) т.е.
	и в этом случае \(\|\alpha\vee\beta\|<\|\gamma\vee\delta\|\).

	Итак, было установлено, что \(\|\alpha\vee\beta\|<\|\gamma\vee\delta\|\), т.е.
	\(x^{\alpha}x^{\beta}<x^{\gamma}x^{\delta}\).

\end{proof}

Для любого кода \(C\) его ортогональное дополнение как линейного подпространства будем называть \emph{дуальным} или \emph{ортогональным} кодом к коду \(C\) и обозначать его символом \(C^{\perp}\).

Дуальным к коду \(RM(r,m)\) является код \(RM(m-r-1,m)\)~\cite{mcwilliams1979theory}.

\emph{Стандартной формой} порождающей матрицы кода Рида-Маллера будем называть матрицу, составленную из векторов значений функций~(\ref{BasesRM}), стоящих в порядке возрастания мономов, соответствующих этим функциям.

Обозначим также символом \(\mathbf{A}(r,m)\) упорядоченное множество всех  таких наборов \(\alpha=(\alpha_{1},\ldots,\alpha_{m})\), что моном \(x^\alpha\) входит в стандартную форму порождающей матрицы кода \(RM(r,m)\).

Приведем пример \(RM(2,4)\):
\begin{displaymath}
	\left(\begin{array}{cccc}
		f      &   & x^\alpha      & \|\alpha\| \\
		\hline
		1      & = & x^{(0,0,0,0)} & 0          \\
		x_4    & = & x^{(0,0,0,1)} & 1          \\
		x_3    & = & x^{(0,0,1,0)} & 2          \\
		x_2    & = & x^{(0,1,0,0)} & 4          \\
		x_1    & = & x^{(1,0,0,0)} & 8          \\
		x_3x_4 & = & x^{(0,0,1,1)} & 3          \\
		x_2x_4 & = & x^{(0,1,0,1)} & 5          \\
		x_2x_3 & = & x^{(0,1,1,0)} & 6          \\
		x_1x_4 & = & x^{(1,0,0,1)} & 9          \\
		x_1x_3 & = & x^{(1,0,1,0)} & 10         \\
		x_1x_2 & = & x^{(1,1,0,0)} & 12         \\
	\end{array}\right)
\end{displaymath}
Причем \(\mathbf{A}(2,4)=\{0,1,2,3,4,5,6,8,9,10,12\}\).

Важнейшую роль в исследованиях кодовых криптосистем играют автоморфизмы кодов.
Пусть \(\sigma\)~--- некоторая подстановка из \(S_n\), а \(v=(v_{1},\ldots,v_n)\) некоторый двоичный вектор.
Тогда через \(v^{\sigma}\) будем обозначать действие подстановки \(\sigma\) на вектор \(v\), т.е.
\(v^{\sigma}=(v_{\sigma(1)},\ldots,v_{\sigma(n)}).\) С подстановкой \(\sigma\in S_n\) можно однозначно связать двоичную перестановочную (\(n\times n\))-матрицу \(P_{\sigma}=(p_{ij})\).
У этой матрицы в каждом столбце и в каждом строке стоит ровно одна единица, причем \(p_{ji}=1\), если и только если \(\sigma(i)=j\).
В этих терминах применение к вектору \(v\) подстановки \(\sigma\) эквивалентно матричному умножению вектора \(v\) на матрицу \(P_\sigma\), т.е.
\(v^\sigma=v\cdot P_\sigma\).
Аналогично можно ввести действие подстановки на некоторое множество векторов.
Если \(C\)~--- некоторое множество двоичных векторов, то \(C^\sigma\)~--- множество, состоящее из всех векторов \(c^\sigma\), где \(c\in C\).
В случае когда \(C\)~--- код, то \(C^\sigma\) также будет являться кодом.
При этом коды \(C\) и \(C^\sigma\) называются \emph{эквивалентными}.

\begin{definition}
	\emph{Автоморфизмом} некоторого линейного \([n,k]\)-кода \(C\) называется подстановка \(\sigma\in S_n\) такая, что \(C^\sigma=C\).
	Множество всех автоморфизмов кода образует группу, которая называется \emph{группой автоморфизмов} кода и обозначается как \( Aut(C).\)
\end{definition}

Зачастую структура группы автоморфизмов кода неизвестна.
Однако в случае кода Рида--Маллера ее можно полностью описать.
Код \(RM(r,m)\) состоит из булевых функций $f(x_1,\ldots, x_m)$.
Пусть \(A\)~--- произвольная невырожденная двоичная \((m\times m)\)-матрица, а \(b=(b_{1},b_{2},\ldots,b_m)\)~---  произвольный двоичный вектор.
Тогда, если $x=(x_1,\ldots, x_m)$, рассмотрим аффинное отображение:
\begin{displaymath}
	\sigma_{A,b}(x)=x\cdot A \oplus b.
\end{displaymath}
Очевидно, что под действием этого отображения функция $f$ переходит в функцию $f(xA\oplus b)$.
Отображение \(\sigma_{A,b} \), в силу невырожденности матрицы $A$, является подстановкой на всех двоичных векторах длины $m$, а так как каждому такому вектору можно однозначно сопоставить целые числа от 0 до $n=2^m-1$, то это отображение является подстановкой из $S_{2^m}$.

Описанное множество подстановок называется \emph{полной аффинной группой} и обозначается как \(GA(m).\) Справедлива следующая теорема.

\begin{proposition}[см.~\cite{mcwilliams1979theory}, стр.~389, теорема~24]
	\label{th1}
	Группа автоморфизмов кода Рида--Маллера \(RM(r,m)\) при \(0<r<m-1\) совпадает с полной аффинной группой, т.е.
	\(Aut(RM(r,m))=GA(m).\)
\end{proposition}

В дальнейшем нам также потребуется фундаментальный результат Р.~Дж.~Мак-Элиса о весовом спектре кода Рида--Маллера.

\begin{proposition}[см.~\cite{mcwilliams1979theory}, стр.~431, следствие~13]
	\label{th_mceliece}
	Вес каждого кодового слова кода Рида--Маллера \(RM(r,m)\) делится на
	\[2^{\lceil \frac{m-1}{r}\rceil}.\]
\end{proposition}

\section {Классификация произведений Адамара подкодов коразмерности 1 кода Рида--Маллера}

\emph{Произведением Адамара} двух векторов будем называть вектор, полученный в результате покомпонентного произведения координат этих векторов:
\[
	(c_1,\ldots, c_n)\circ (b_1,\ldots, b_n)=(c_1b_1,\ldots, c_nb_n).

\]

\begin{definition}
	Пусть \(C\) и \(B\)~--- два линейных \([n,k_1]\) и \([n,k_2]\)-кода соответственно и пусть \(\{c_1,\ldots, c_{k_1}\}\)~--- базис кода $C$, а \(\{b_1,\ldots,b_{k_2}\}\)~--- базис кода \(B\).
	Тогда произведением Адамара \(C\circ B\) кодов \(C\) и \(B\) будем называть код, состоящий из векторов линейной оболочки множества \(\{c_i\circ b_j|1\leqslant i\leqslant k_1,\ 1\leqslant j\leqslant k_2\}\).
\end{definition}

В дальнейшем для натурального \(q\) будем обозначать символом \(qC\) код
\begin{displaymath}
	qC=\underbrace{C\circ C\circ\ldots\circ C}_{q}.
\end{displaymath}

Итак, пусть \(C\)~--- произвольный \((k-1)\)-подкод кода Рида--Маллера \(RM(r,m)\).
Рассмотрим такой моном \(x_{min}=x^{\alpha_{min}}\), что для всех \(\beta<\alpha_{min}\) моном \(x^{\beta}\in C\), а \(x_{min}\not\in C\).
Заметим, что такой моном существует, так как подкод \(C\) не совпадает со всем кодом \(RM(r,m),\) и находится методом перебора всех мономов, начиная с монома \(1=x^{(0,0,\ldots,0)}\).
Для всех \(\beta>\alpha_{min}\) либо моном \(x^{\beta}\in C\), либо \(x^{\beta}\oplus x^{\alpha_{min}} \in C\), т.е.
\(x^{\beta}\oplus a_{\beta}x^{\alpha_{min}}\in C\) для некоторого \(a_{\beta}\in\{0,1\}.\) Значит можно ввести вектор \(a=(a(\beta)=a_{\beta}|\beta\in \mathbf{A}(r,m)),\) который совместно с \(\alpha_{min}\) определяет подкод \(C\).
Будем в дальнейшем такой подкод обозначать как \(C_{\alpha_{min},a}(r,m)\), причем \(a(\beta)=0\) для всех \(\beta<\alpha_{min}\).

Обратимся теперь к вопросу как устроен квадрат Адамара кода \(C_{\alpha, a}(r,m)\).
Следующие две теоремы описывают классификацию квадрата Адамара таких кодов.

%%%%%%%%%%%%%%%%%%%%%%%%%%%%%%%%%%%%%%%%%%%%%%%%%%%%%%%
%Теорема о произведении подкодов
%%%%%%%%%%%%%%%%%%%%%%%%%%%%%%%%%%%%%%%%%%%%%%%%%%%%%%%

\begin{theorem}
	\label{SquareOfAlpha11}
	Пусть \(r_1+r_2 < m\).
	Пусть также \(\alpha^1\in\mathbf{A}(r_1,m), \alpha^2\in \mathbf{A}(r_2,m)\), причем выполнено одно из двух условий:
	\begin{itemize}

		\item[1)]\(\alpha^1\neq\alpha^2,\ \alpha^1,\alpha^2>0;\)
		\item[2)]\(\alpha^1=\alpha^2=\alpha\) и \(\wt(\alpha)\geqslant 2\).
	\end{itemize}
	Тогда для любых \(a^1\) и \(a^2\) выполняется равенство:
	\begin{equation}
		\label{CC_R}
		C_{\alpha^1,a^1}(r_1,m)\circ C_{\alpha^2,a^2}(r_2,m)=RM(r_1+r_2,m).
	\end{equation}
\end{theorem}
\begin{proof}
	Для доказательства теоремы нам потребуется следующая лемма о существовании разбиения некоторого множества на два подмножества специального вида.

	%%%%%%%%%%%%%%%%%%%%
	%Лемма о разбиении
	%%%%%%%%%%%%%%%%%%%%
	\begin{lemma}[О разбиении]
		\label{Utv_Split}
		В условиях теоремы для любого подмножества $M \subset \{1,2,...,m\}$, для которого $1 \le |M|\le r_1+r_2$, существуют такие подмножества $M_1,M_2\subseteq M$, что:
		\begin{itemize}
			\item[1)] \(M_1\cup M_2 = M\) и \(M_1\cap M_2 = \emptyset,\) т.е.
			      подмножества \(M_1\) и \(M_2\) образуют \emph{разбиение} множества \(M\);
			\item[2)] $|M_1|\leqslant r_1$ и \(|M_2|\leqslant r_2\);
			\item[3)] $M_1 \neq \supp(\alpha^1)$ и \(M_2 \neq \supp(\alpha^2)\).
		\end{itemize}
	\end{lemma}

	\begin{proof}
		Для начала заметим, что в условиях теоремы множества \(\supp(\alpha^1)\) и \(\supp(\alpha^2)\) не пустые и они либо различны, либо совпадают и их мощность при этом по крайней мере 2.

		Пусть $|M| \geqslant 2$ и хотя бы одно из множеств \(\supp(\alpha^1)\) и \(\supp(\alpha^2)\) не лежит целиком в \(M\).
		Тогда, не ограничивая общности, пусть \(\supp(\alpha^1)\not\subseteq M\).
		Выберем произвольное разбиение \(T_1\) и \(T_2\) множества $M$, которое удовлетворяет условию 2) леммы и в котором никакое из множеств не пустое.
		Если $T_2 \neq \supp(\alpha^2)$, то полученное разбиение является искомым, т.к.
		и \(T_1\neq \supp(\alpha^1)\) в силу того, что \(\supp(\alpha^1)\not\subseteq M.\)
		Если же так оказалось, что $T_2 = \supp(\alpha^2)$, то в непустом множестве $T_1$ существует элемент \(a\), который не лежит в \(\supp(\alpha^2)\), т.к.
		все элементы \(\supp(\alpha^2)\) лежат в $T_2$.
		Выберем в \(T_2\) любой элемент \(b\) и построим новое разбиение множества $M$, поменяв элементы \(a\) и \(b\) между множествами \(T_1\) и \(T_2\): $M_1 = (T_1 \setminus \{a\}) \cup \{b\}$, $M_2 = (T_2 \setminus \{b\}) \cup \{a\}$.
		Очевидно, что для разбиения $M_1$ и $M_2$ условие 2) леммы не нарушится.
		При этом по построению $M_2 \neq \supp(\alpha^2)$, а т.к.
		$\supp(\alpha^1) \not\subseteq M$, то и $M_1 \neq \supp(\alpha^1)$, значит и условие 3) выполнено.

		Если $\supp(\alpha^1) \cup \supp(\alpha^2) \subseteq M$, то здесь необходимо рассмотреть два варианта.

		Первый вариант~--- $\supp(\alpha^1) \neq \supp(\alpha^2)$.
		Существует такой элемент $a$, который не лежит в \(\supp(\alpha^1)\), но лежит в \(\supp(\alpha^2)\).
		Тогда разбиение множества $M$ на $M_1$ и $M_2$ следует выбрать так, чтобы, во-первых, оно удовлетворяло условию 2), а во-вторых, чтобы $a \in M_1$.
		При таком разбиении $a\not\in M_2$, значит $M_2 \neq \supp(\alpha^2)$ и $M_1 \neq \supp(\alpha^1)$, т.к.
		в \(M_1\) нет элемента \(a\) из \(\supp(\alpha^1)\).
		Получаем, что для такого разбиения $M_1$ и $M_2$ выполнено и условие 3).

		Второй вариант~--- $\supp(\alpha^1) = \supp(\alpha^2)$.
		Тогда, по условию теоремы, мощности каждого из этих множеств не менее 2.
		Это означает, что найдутся как минимум два различных элемента, которые принадлежат $\supp(\alpha^1)$, обозначим их как $a$ и $b$.
		Разбиение множества $M$ на $M_1$ и $M_2$, следует выбрать так, чтобы, во-первых, оно удовлетворяло условию 2), а во-вторых, $a \in M_1$ и $b \in M_2$.
		Так как $b \in \supp(\alpha^1)$ и $b \not\in M_1$, то верно что $M_1 \neq \supp(\alpha^1)$.
		Аналогично, из того, что $a \in \supp(\alpha^2)$ и $a \not\in M_2$, следует, что $M_2 \neq \supp(\alpha^2)$.
		Получаем, что для разбиения $M_1$ и $M_2$ выполнено также условие 3).

		И, наконец, рассмотрим случай $|M| = 1$.
		Пусть $M=\{a\}$.
		Если $\supp(\alpha^1) = \supp(\alpha^2)$, то согласно условию теоремы, $|\supp(\alpha^1)| \ge 2$, и тогда разбиение $M_1 = \{a\}, M_2 = \emptyset$ является искомым.
		А когда $\supp(\alpha^1) \neq \supp(\alpha^2)$, эти два множества не могут одновременно равняться $\{a\}$.
		Не ограничивая общности, будем считать, что $\supp(\alpha^1) \neq \{a\}$, тогда разбиение  $M_1 = \{a\}, M_2 = \emptyset$ также является искомым.
	\end{proof}

	Перейдем к доказательству теоремы.

	Равенство~(\ref{CC_R}) выполняется тогда и только тогда, когда для дуального кода справедливо:
	$\left( C_{\alpha^1,a^1}(r_1,m)\circ C_{\alpha^2,a^2}(r_2,m)\right)^{\perp}=RM(m-r_1-r_2-1,m).$
	Вложение кода $RM(m-r_1-r_2-1,m)$ в код $( C_{\alpha^1,a^1}(r_1,m)\circ C_{\alpha^2,a^2}(r_2,m))^{\perp}$ следует из определения произведения Адамара двух кодов и из свойств кода Рида--Маллера.
	Докажем, что справедливо обратное вложение, доказательство проведем <<от противного>>.

	Предположим, что найдётся такой вектор $f$, который не лежит в $RM(m-r_1-r_2-1,m)$, но лежит в $\left( C_{\alpha^1,a^1}(r_1,m)\circ C_{\alpha^2,a^2}(r_2,m)\right)^{\perp}$.
	Пусть \(x^{\lambda}\) ~--- старший моном относительно введённого порядка в его полиноме Жегалкина, т.е.
	для любого другого монома \(x^{\gamma}\) из полинома Жегалкина выполняется неравенство \(x^{\gamma}<x^{\lambda}\).

	Так как $f \not\in RM(m-r_1-r_2-1,m)$, то $\deg(f)\geqslant m-r_1-r_2$, а значит \(|\supp(\lambda)|\geqslant m - r_1 - r_2\).
	Рассмотрим множество:
	\[
		M_{\lambda} = \{1,2,\ldots,m\}\setminus\supp(\lambda).

	\]

	Причём \(|M_{\lambda}|\leqslant r_1 + r_2\).
	Из леммы \ref{Utv_Split} следует, что можно выбрать такие $\beta^1$ и $\beta^2$, что $\supp(\beta^1)$ и $\supp(\beta^2)$ не пересекаются, в объединении дают всё множество $M_{\lambda}$ и при этом $x^{\beta^1}\oplus a^1(\beta^1)x^{\alpha^1}$ лежит в коде $C_{\alpha^1,a^1}(r_1,m)$, а $x^{\beta^2} \oplus a^2(\beta^2)x^{\alpha^2}$~--- в коде $C_{\alpha^2,a^2}(r_2,m)$.
	Далее будет доказано, что вектор
	\[y_{\beta^1,\beta^2}=(x^{\beta^1}\oplus a^1(\beta^1)x^{\alpha^1})(x^{\beta^2} \oplus a^2(\beta^2)x^{\alpha^2})\in C_{\alpha^1,a^1}(r_1,m)\circ C_{\alpha^2,a^2}(r_2,m)\]
	не может быть ортогонален вектору \(f\), что будет означать противоречие с существованием последнего.
	Для этого достаточно установить, что \(y_{\beta^1,\beta^2}\cdot f\) как булева функция имеет максимально возможную степень.
	Действительно, скалярное произведение векторов-значений булевых функций может быть выражено через сумму по модулю 2 всех координат вектора-значений произведения этих булевых функций.
	Таким образом, если вес Хэмминга вектора-значений произведения функций нечётен, то такие функции не могут быть ортогональны.
	Но известно~\cite{mcwilliams1979theory}, что все функции чётного веса лежат в коде Рида--Маллера \(RM(m-1,m)\), а значит, если произведение булевых функций не будет лежать в этом коде, т.е.
	будет иметь максимально возможную степень \(m\), то его вес будет нечётным.

	Чтобы установить, что булева функция \(y_{\beta^1,\beta^2}\cdot f\) имеет максимально возможную степень, будет далее доказано, что для максимального монома \(x^{\lambda}\) из полинома Жегалкина функции \(f\) функция \(y_{\beta^1,\beta^2}\cdot x^{\lambda}\) будет иметь максимальную степень, а для всех остальных мономов \(x^{\gamma}\) (они будут строго меньше \(x^{\lambda}\)) из этого полинома функция \(y_{\beta^1,\beta^2}\cdot x^{\gamma}\) будет иметь степень строго меньшую \(m\).

	По определению \(M_{\lambda}\), а также по построению \(\beta^1\) и \(\beta^2\), множества \(\supp(\beta^1)\), \(\supp(\beta^2)\) и \(\supp(\lambda)\) попарно не пересекаются и в объединении дают множество $\{1,2,...,m\}$, а значит
	\begin{equation}
		\label{eq_maxmonom}
		x^{\beta^1}x^{\beta^2}x^{\lambda}=x_1\ldots x_m.
	\end{equation}

	Пусть \(x^{\gamma}\)~--- произвольный моном.
	Рассмотрим произведение
	\begin{equation}
		\label{eq_prod}
		\left( \left(x^{\beta^1}\oplus a^1(\beta^1)x^{\alpha^1}\right)\cdot \left(x^{\beta^2} \oplus a^2(\beta^2)x^{\alpha^2}\right)\right)\cdot x^{\gamma}
	\end{equation}
	Раскрывая скобки, получим
	\begin{equation}
		\label{eq_sum_monoms}
		x^{\beta^1}x^{\beta^2}x^{\gamma}\oplus a^1(\beta^1)x^{\alpha^1}x^{\beta^2}x^{\gamma} \oplus a^2(\beta^2)x^{\beta^1}x^{\alpha^2}x^{\gamma} \oplus a^1(\beta^1)a^2(\beta^2)x^{\alpha^1}x^{\alpha^2}x^{\gamma}
	\end{equation}

	С учётом построения кодов \(C_{\alpha^i,a^i}(r_i,m),\) \(i=1,2\), если сумма~(\ref{eq_sum_monoms}) содержит моном \(x^{\alpha^1}x^{\beta^2}x^{\gamma}\), то \(x^{\alpha^1}<x^{\beta^1}\), если~--- моном \(x^{\beta^1}x^{\alpha^2}x^{\gamma}\), то \(x^{\alpha^2}<x^{\beta^2}\) и, наконец, если~--- моном \(x^{\alpha^1}x^{\alpha^2}x^{\gamma}\), то одновременно \(x^{\alpha^1}<x^{\beta^1}\) и \(x^{\alpha^2}<x^{\beta^2}\).

	Рассмотрим две ситуации: \(\gamma=\lambda\) и \(\gamma < \lambda\).

	Итак, пусть \(\gamma=\lambda\).
	Тогда в силу~(\ref{eq_maxmonom}) моном \(x^{\beta^1}x^{\beta^2}x^{\lambda}\) будет иметь максимально возможную степень.
	Если \(x^{\alpha^1}<x^{\beta^1}\), то в силу утверждения~\ref{prop_order} с учётом того, что \(\beta^1\), \(\beta^2\) и \(\lambda\) попарно не пересекаются, моном \(x^{\alpha^1}x^{\beta^2}x^{\lambda}\) строго меньше монома \(x^{\beta^1}x^{\beta^2}x^{\lambda}\), который имеет максимальную степень.
	Значит степень монома \(x^{\alpha^1}x^{\beta^2}x^{\lambda}\) строго меньше \(m\).
	Аналогично, основываясь на неравенстве \(x^{\alpha^2}<x^{\beta^2}\), можно доказать, что моном \(x^{\beta^1}x^{\alpha^2}x^{\lambda}\) также не может иметь максимальную степень.
	И, наконец, с учётом неравенств \(x^{\alpha^1}<x^{\beta^1}\) и \(x^{\alpha^2}<x^{\beta^2}\), используя утверждения~\ref{prop_order}, можно установить, что моном также не может иметь максимальную степень.
	Таким образом, в этом случае произведение~(\ref{eq_prod}) имеет степень в точности равную \(m\).

	В случае, когда \(\gamma < \lambda\), используя утверждение~\ref{prop_order}, формулу~(\ref{eq_maxmonom}), на основе неравенств \(x^{\gamma}<x^{\lambda},\) \(x^{\alpha^1}<x^{\beta^1}\) и \(x^{\alpha^2}<x^{\beta^2}\) можно доказать, что все мономы в сумме~(\ref{eq_sum_monoms}) имеют степень строго меньшую \(m\).
	А значит в этом случае произведение~(\ref{eq_prod}) имеет степень строго меньшую \(m\).

	Итак, в конечном итоге доказано, что произведение функций \(x^{\beta^1}\oplus a^1(\beta^1)x^{\alpha^1}\), \(x^{\beta^2} \oplus a^2(\beta^2)x^{\alpha^2}\) и \(f\) имеет максимально возможную степень, т.е.
	получили противоречие с тем, что \(f\) принадлежит коду, дуальному к коду \(C_{\alpha^1,a^1}(r_1,m)\circ C_{\alpha^2,a^2}(r_2,m)\).
\end{proof}

%%%%%%%%%%%%%%%%%%%%%%%%%%%%%%%%%%%%%%%%%%%%%%%%%%%%%%%%
Из теоремы~\ref{SquareOfAlpha11} можно получить полезное в дальнейшем следствие

\begin{corollary}
	\label{th_rm_c_alpha}
	Пусть \(r_1+r_2<m\).
	Тогда для любого \(\alpha\in\mathbf{A}(r_1,m), \alpha\neq 0,\) и любого \(a\) справедливо равенство
	\begin{displaymath}
		C_{\alpha,a}(r_1,m)\circ RM(r_2,m)=RM(r_1+r_2,m).
	\end{displaymath}
\end{corollary}
\begin{proof}
	Выберем в коде \(RM(r_2,m)\) любой такой подкод \(C_{\beta, 0}(r_2,m),\) что \(\beta\neq\alpha\) и \(\beta \neq 0\).
	Тогда согласно теореме~\ref{SquareOfAlpha11} выполняется равенство
	\(C_{\alpha, a}(r_1,m)\circ C_{\beta, 0} = RM(r_1+r_{2}, m).\) Получается, что \(RM(r_{1}+r_{2}, m)\subseteq C_{\alpha,a}(r_1,m)\circ RM(r_2,m).\) Обратное включение очевидно.
\end{proof}
%%%%%%%%%%%%%%%%%%%%%%%%%%%%%%%%%%%%%%%%%%%%%%%%%%%%%%%%

\begin{theorem}
	\label{SquareOfAlpha2}
	Пусть \(2r<m\).
	Тогда для любых \(\alpha \text{ таких, что} \wt(\alpha)=1,\) справедливо либо
	\begin{displaymath}
		C_{\alpha,a}(r,m)\circ C_{\alpha,a}(r,m)=RM(2r,m),
	\end{displaymath}
	либо существует такой автоморфизм \(\sigma_{A,b}\) кода Рида--Маллера \(RM(r,m),\) что
	\begin{displaymath}
		C_{\alpha,a}(r,m)\circ C_{\alpha,a}(r,m)=C_{1,0}^{\sigma_{A,b}}(2r,m).
	\end{displaymath}
\end{theorem}
\begin{proof}
	Проведем доказательство по следующему плану:
	\begin{enumerate}
		\item Для кода $C_{\alpha,a}(r,m)$ построим такой автоморфизм $\sigma_{S,0}$, что $x^{\beta} \in \Bigl(C_{\alpha,a}(r,m)\Bigr)^{\sigma_{S,0}}$, для всех $\beta \not= \alpha$ и $|\supp(\beta)|= 1$.
		\item Построим такой автоморфизм $\sigma_{E,b}$ (здесь \(E\)~--- единичная матрица), что $x^{\alpha}x^{\beta} \in \Bigr((C_{\alpha,a}(r,m))^{\sigma_{A,0}}\Bigr)^{\sigma_{E,b}}$, для всех $\beta \not= \alpha$ и $|\supp(\beta)| = 1 $.
		\item Покажем, что произведение кодов $C_{\alpha,a}(r,m)\circ C_{\alpha,a}(r,m)$ равно либо $RM(2r,m)$, либо $C_{\alpha,0}^{\sigma_{S,b}}(2r,m)$, где $\sigma_{S,b} = \sigma_{S,0} \cdot \sigma_{E,b}.$
		\item Если произведение кодов равно $C_{\alpha,0}^{\sigma_{S,b}}(2r,m)$, то применим такой автоморфизм $\sigma$, который осуществляет переименование переменных, т.е.
		      переводит \(x^{\alpha} \longleftrightarrow x^{1}\).
		      Композиция $\sigma\cdot \sigma_{S,0}\cdot \sigma_{E,b}$ и будет искомым автоморфизмом $\sigma_{A,b}$.
	\end{enumerate}

	%%%%%%%%%%%%%%%%%%%%
	%Первый шаг
	%%%%%%%%%%%%%%%%%%%%
	\begin{lemma}
		\label{Lem_Auth_1}
		Для всех $\beta \not= \alpha$ и $|\supp(\beta)| = 1$ мономы $x^{\beta}$ принадлежат коду $\Bigl(C_{\alpha,a}(r,m)\Bigr)^{\sigma_{S,0}}$, где автоморфизм $\sigma_{S,0}$ определяется следующим образом: для таких $\gamma$, что $|\supp(\gamma)| = 1$:
		\begin{displaymath}
			x^{\gamma} \xrightarrow
			[]{\sigma_{A,0}}
			\begin{cases}
				x^{\gamma}\oplus a(\gamma)x^{\alpha}, & \gamma \neq \alpha, \\
				x^{\alpha},                           & \gamma = \alpha.
				\\
			\end{cases}
		\end{displaymath}
	\end{lemma}
	\begin{proof}
		Следует из описания автоморфизма.
	\end{proof}
	Под действием автоморфизма $\sigma_{S,0}$ код $C_{\alpha,a}(r,m)$ преобразуется в код $C_{\alpha,a'}(r,m)$.
	При этом мономы $$x^{\alpha}x^{\beta} \oplus a(\alpha \vee \beta )x^{\alpha} \xrightarrow[]{\sigma_{S,0}} x^{\alpha}x^{\beta} \oplus (a(\alpha \vee \beta )\oplus a(\beta))x^{\alpha}, \text{ \ \ для } \beta \ne \alpha \text{ и } |\supp(\beta)|=1.$$
	Это означает, что для указанных $\beta$, будет верно равенство $a'(\alpha \vee \beta) =  a(\alpha \vee \beta)\oplus a(\beta)$.
	Используем это, для построения автоморфизма, согласно второму шагу плана доказательства.
	%%%%%%%%%%%%%%%%%%%%
	%Второй шаг
	%%%%%%%%%%%%%%%%%%%%
	\begin{lemma}
		\label{Lem_Auth_2}
		Для всех $\beta \ne \alpha$ и $|\supp(\beta)|=1$ мономы $x^{\alpha}x^{\beta}$ принадлежат коду $(C_{\alpha,a'}(r,m))^{\sigma_{E,b}}$, где автоморфизм $\sigma_{E,b}$ определяется следующим образом: для таких $\gamma$, что $|\supp(\gamma)| = 1$:
		\begin{displaymath}
			x^{\gamma} \xrightarrow []{\sigma_{E,b}}
			\begin{cases}
				x^{\gamma} \oplus a'(\alpha \vee \gamma), & \gamma \neq \alpha, \\
				x^\alpha,                                 & \gamma = \alpha.
				\\
			\end{cases}
		\end{displaymath}
	\end{lemma}
	\begin{proof}
		Следует из описания автоморфизма.
	\end{proof}

	Под действием автоморфизма $\sigma_{E,b}$ код $C_{\alpha,a'}(r,m)$ переходит в код $C_{\alpha,a''}(r,m)$ для некоторого вектора $a''$.
	Изучим код $(C_{\alpha,a''}(r,m))^2 = C_{\alpha,a''}(r,m) \circ C_{\alpha,a''}(r,m)$.
	%%%%%%%%%%%%%%%%%%%%
	%Критерий
	%%%%%%%%%%%%%%%%%%%%
	\begin{lemma}
		\label{Lem_Auth_3}
		\begin{displaymath}
			(C_{\alpha,a''}(r,m))^2 =
			\begin{cases}
				RM(2r,m),               & \text{если } x^{\alpha} \in (C_{\alpha,a''}(r,m))^2, \\
				C_{x^{\alpha},0}(2r,m), & \text{иначе.}                                        \\
			\end{cases}.
		\end{displaymath}
	\end{lemma}
	\begin{proof}
		Доказательство в обоих случаях проведем по индукции.

		Случай $x^{\alpha} \in (C_{\alpha,a''}(r,m))^2$.

		База индукции.
		Из леммы~\ref{Lem_Auth_1} следует, что все мономы первой степени, кроме, быть может, $x^{\alpha}$ принадлежат коду $C_{\alpha,a''}(r,m)$.
		Из леммы \ref{Lem_Auth_2} следует, что все мономы вида $x^{\alpha}x^{\beta}$ принадлежат $C_{\alpha,a''}(r,m),$ где $|\supp(\beta)|=1$, $\beta \not=\alpha$.
		Принимая во внимание построение кода $(C_{\alpha,a''}(r,m))^2$ и то, что в этом коде лежит вектор $x^{\alpha},$ получаем, что $RM(2,m)\subseteq (C_{\alpha,a''}(r,m))^2.$

		Предположение индукции.
		Пусть $RM(t,m)\subset (C_{\alpha,a''}(r,m))^2,$ где $2 \leqslant t \leqslant 2r-1$.

		Следует доказать, что $RM(t+1,m)$ принадлежит $(C_{\alpha,a''}(r,m))^2$.
		Учитывая предположение, достаточно показать, что все мономы степени $t+1$ принадлежат коду $(C_{\alpha,a''}(r,m))^2$.
		Рассмотрим произвольный моном $x^{\gamma}$, где $|\supp(\gamma)| = t+1$.
		Коду $C_{\alpha,a''}(r,m)$ принадлежат слова $x^{\gamma_1} \oplus a''(\gamma_1)x^{\alpha}$ и $x^{\gamma_2} \oplus a''(\gamma_2)x^{\alpha}$, где $|\supp(\gamma_1)| = \lfloor\frac{t+1}{2}\rfloor,$ $|\supp(\gamma_2)| = t + 1 - |\supp(\gamma_1)|$ и $\supp(\gamma) = \supp(\gamma_1) \cup \supp(\gamma_2)$.
		В результате умножения этих векторов получится:
		\begin{equation*}
			\begin{split}
				(x^{\gamma_1} \oplus a''(\gamma_1)x^{\alpha} )\cdot( x^{\gamma_2} \oplus a''(\gamma_2)x^{\alpha} )&= x^k \oplus a''(\gamma_2)x^{\gamma_1}x^{\alpha} \oplus a''(\gamma_1)x^{\gamma_2}x^{\alpha} \oplus\\
				&\oplus a''(\gamma_1)a''(\gamma_2)x^{\alpha}.

			\end{split}
		\end{equation*}

		Заметим, что степень мономов $x^{\gamma_1}x^{\alpha}$, $x^{\gamma_2}x^{\alpha}$ и $x^{\alpha}$ не больше $t$, значит, согласно предположению индукции, они принадлежат коду $(C_{\alpha,a''}(r,m))^2$.
		Получаем, что моном $x^\gamma$ принадлежит $(C_{\alpha,a''}(r,m))^2$.
		В силу произвольности выбора $x^\gamma$ справедливо включение $RM(t+1,m) \subseteq (C_{\alpha,a''}(r,m))^2$

		Случай $x^{\alpha} \not\in (C_{\alpha,a''}(r,m))^2$.
		Докажем индукцией по степени мономов, что $(C_{\alpha,a''}(r,m))^2=C_{\alpha,0}(2r,m)$.

		База индукции.
		Согласно лемме~\ref{Lem_Auth_2}, все мономы $x^{\alpha}x^{\beta} \in C_{\alpha,a''}(r,m),$ где $\beta \not= \alpha,$ $|\supp(\beta)|=1.$
		Согласно лемме~\ref{Lem_Auth_1} мономы $x^{\beta} \in C_{\alpha,a''}(r,m),$ где $\beta \not= \alpha,$ $ |\supp(\beta)|=1$, значит, согласно построению кода $(C_{\alpha,a''}(r,m))^2$, мономы вида $x^{\beta}x^{\gamma},$ где $\gamma \not= \alpha$, $|\supp(\gamma)|=1$,  принадлежат $(C_{\alpha,a''}(r,m))^2.$ Значит $C_{\alpha,0}(2,m)\subseteq (C_{\alpha,a''}(r,m))^2$.

		Предположение индукции.
		Пусть все мономы степени не больше $2\leqslant t\leqslant 2r-1$ принадлежат $(C_{\alpha,a''}(r,m))^2$, кроме монома $x^{\alpha}$.

		Следует доказать, что все мономы степени $t+1$ также принадлежат этому коду.
		Заметим, что из условия $x^{\alpha} \not\in (C_{\alpha,a''}(r,m))^2$ и предположения индукции следует, что все мономы степени не более $t$, кроме монома $x^{\alpha}$, принадлежат квадрату этого кода.

		Не ограничивая общности, рассмотрим моном $x^{\gamma}$, где $|\supp(\gamma)| = t+1$.
		Коду $C_{\alpha,a''}(r,m)$ принадлежат кодовые слова $x^{\gamma_1} \oplus a''(\gamma_1)x^{\alpha}$ и $x^{\gamma_2} \oplus a''(\gamma_2)x^{\alpha}$, где $|\supp(\gamma_1)| = \lfloor\frac{t+1}{2}\rfloor,$ и $|\supp(\gamma_2)| = t+1 - |\supp(\gamma_1)|$.
		Эти же кодовые слова принадлежат коду $(C_{\alpha,a''}(r,m))^2$.
		Отсюда следует, что моном $x^{\gamma}$ принадлежит $(C_{\alpha,a''}(r,m))^2$.
	\end{proof}

	Применим теперь к коду $(C_{\alpha,a''}(r,m))^2=C_{\alpha,0}(2r,m)$ автоморфизм $\sigma$, который переведет моном $x^{\alpha}$ в моном $x^1$.
	Учитывая лемму \ref{Lem_Auth_3}, получаем, что $((C_{\alpha,a''}(r,m))^2)^{\sigma}$ равен либо $RM(2r,m)$, либо $C_{1,0}(2r,m).$
\end{proof}

\section{Криптосистемы Мак-Элиса, построенные на подкодах кода Рида--Маллера}
В работах~\cite{minder2007cryptanalysis,borodin2014effective} предложены достаточно эффективные атаки на криптосистему Мак-Элиса, построенную на основе кодов Рида--Маллера.
Поэтому по аналогии с криптосистемами, основанными на подкодах кода Рида--Соломона~\cite{berger2005how}, можно рассмотреть криптосистему Мак-Элиса на подкодах Рида--Маллера в надежде, что это повысит стойкость такой криптосистемы.

\subsection{Криптосистема первого типа}
Для генерации ключей строится стандартная форма порождающей матрицы \(R\) кода Рида--Маллера $RM(r,m) $.
Далее выбирается случайная двоичная невырожденная \((k\times k)\)-матрица \(H=(h_{ij})\) и случайная подстановка \(\sigma\in S_n\), представленная в виде перестановочной \((n\times n)\)-матрицы \(P_\sigma\).
Затем, вычисляется матрица \(G'=H\cdot R\cdot P_{\sigma}=H\cdot R^{ \sigma}\) и из неё удаляется первая строка, получается \(((k-1)\times  n)\)-матрица \(G\).
\emph{Секретным ключом} криптосистемы является набор --- \((H,P_{\sigma})\) или набор \((H,\sigma)\), а \emph{открытым ключом} является матрица \(G\) и \((r,m)\)~--- параметры кода Рида--Маллера, однако, ради удобства, параметры в открытый ключ не включены.
Матрица \(G\)  порождает некоторый \((k-1)\)-подкод кода Рида--Маллера \(RM(r,m)\), <<испорченный>> подстановкой \(\sigma\).
Криптосистему с параметрами $r$ и $m$ будем обозначать через $McElSubRM(r,m)$.

Сделаем некоторое замечание по поводу удаления первой строки из матрицы \(G'\).
Можно удалять из этой матрицы любую строку с номером \(i\).
Однако, этот процесс может быть сведён к удалению первой строки, если вместо матрицы \(H\) выбрать матрицу \(\Gamma_{i\leftrightarrow 1} \cdot H\), где \(\Gamma_{i\leftrightarrow 1}\)~--- перестановочная \((k\times k)-\)матрица, соответствующая транспозиции строк с номерами $i$ и $1$.

\begin{definition}
	Два секретных ключа \((H_1,\sigma_1) \  \) и \((H_2, \sigma_2 )\) называются \emph{эквивалентными}, если соответствующие им открытые ключи \(G_1\) и \(G_2\) равны.
\end{definition}

Пусть \((H,\sigma)\)~--- секретный ключ криптосистемы \(McElSubRM(r,m)\), \(G\)~--- соответствующий ему открытый ключ, и пусть \(\sigma_{A,b}\)~--- некоторый автоморфизм кода Рида--Маллера.
Тогда для порождающей матрицы \(R\) кода Рида--Маллера существует такая единственная матрица \(H_{A,b}\),  что
\begin{equation}
	\label{AutoMat}
	H_{A,b}R=R\sigma_{A,b},
\end{equation}
причем матрица \(H_{A,b}\) является невырожденной.
\begin{theorem}
	Пусть \([(H,\sigma)]\)~--- класс эквивалентности секретного ключа \((H,\sigma)\) криптосистемы \(McElSubRM(r,m)\).
	Тогда
	\begin{displaymath}
		\{(HH_{A,b},\sigma^{-1}_{A,b}\sigma)|\sigma_{A,b}\in Aut(RM(r.m))\}\subseteq [(H,\sigma)].
	\end{displaymath}
\end{theorem}

\begin{proof}
	Из формулы~(\ref{AutoMat}) следует равенство:
	\begin{equation}\label{Equiv1}
		HR\sigma =H\cdot H_{A,b}\cdot R\cdot \sigma_{A,b}^{-1}\cdot \sigma=H_1R\sigma_1,
	\end{equation}
	где \(H_1=H\cdot H_{A,b}\) и \(\sigma_1=\sigma^{-1}_{A,b}\sigma\).
	Удаление из матриц, стоящих в левой и правой частях равенства~(\ref{Equiv1}), первой строки не изменит это равенство.
	Значит, открытый ключ, соответствующий секретному ключу \((H_{1},\sigma_1)\), совпадает с открытым ключом \(G\), т.е.
	ключи \((H,\sigma)\) и \((HH_{A,b},\sigma^{-1}_{A,b}\sigma)\) эквивалентны.
\end{proof}

\subsection{Устройство криптосистемы второго типа}
Для генерации ключей также строится стандартная форма порождающей матрицы \(R\) кода Рида--Маллера \(RM(r,m).\)  Далее выбирается случайный номер \(1\leqslant  i\leqslant k\).
Из матрицы \(R\) выкидывается строка с номером \(i.\) Получившуюся в результате матрицу обозначим через \(R[i]\).
Выбирается случайная двоичная невырожденная \(((k-1)\times (k-1))\)-матрица \(H=(h_{ij})\) и случайная перестановочная (\(n\times n\))-матрица \(P_{\sigma}=(p_{ij})\).
Вычисляется матрица \(G=H\cdot R[i]\cdot P_{\sigma}=H\cdot (R[i])^{\sigma}.\) \emph{Секретным ключом} криптосистемы является набор --- \((H,P_{\sigma},i)=(H,\sigma,i)\), а \emph{открытым ключом} является матрица \(G\).

\begin{definition}
	Два секретных ключа \((H_1,\sigma_1, i_1) \  \) и \((H_2, \sigma_2, i_2 )\) называются \emph{эквивалентными}, если соответствующие им открытые ключи \(G_1\) и \(G_2\) равны.
\end{definition}

\begin{proposition}
	\label{prop_h1_h2_equiv}
	Для любой невырожденной $((k-1)\times(k-1))$-матрицы $H_1$ и любого номера $i$ найдётся такая невырожденная $(k\times k)$-матрица $H_2$, что матрица
	\(H_1R[i]\) получается из матрицы \(H_2R\) удалением первой строки.
\end{proposition}
\begin{proof}
	Построим матрицу $H_2$ из матрицы $H_1$ следующим образом:
	\[
		H_2=\begin{pmatrix}
			0                & \ldots & 0                    & 1 & 0                & \ldots & 0                     \\
			h^{\downarrow}_1 & \ldots & h^{\downarrow}_{i-1} & 0 & h^{\downarrow}_i & \ldots & h^{\downarrow}_{k-1},
		\end{pmatrix}
	\]
	где $h^{\downarrow}_i$~--- строка с номером $i$ матрицы $H_1$.
	Тогда легко убедиться, что, удалением первой строки из матрицы $H_2R$, получается матрица $H_1R[i]$.
\end{proof}
Утверждение~\ref{prop_h1_h2_equiv} позволяет заключить, что открытый ключ криптосистемы второго типа может рассматривать как открытый ключ криптосистемы первого типа, а значит, в этом смысле криптосистема второго типа является частным случаем криптосистемы первого типа и в дальнейшем её можно не рассматривать.

%%%%%%%%%%%%%%%%%%%%%%%%%%%%%%%%%%%%%%%%%%%%%%%%%%%%%%%%
\section{Криптоанализ криптосистемы, построенной на подкодах коразмерности 1}

\subsection{Постановка задачи криптоанализа}

Для криптосистемы \(McElSubRM(r,m)\)  требуется по открытому ключу криптосистемы~--- \(G\)  найти такую пару \((H,\sigma)\), что после удаления первой строки из матрицы \(HR^{\sigma}\) получится матрица \(G.\)

Фактически матрица \(G^{\sigma}\) порождает некоторый подкод кода Рида--Маллера размерности \(k-1\) или коразмерности 1.
По этой причине большая часть работы посвящена исследованию таких подкодов.

Для построения эффективной атаки на криптосистему первого типа будем использовать идеи из работы~\cite{borodin2014effective}.

\subsection{Алгоритм атаки на криптосистему, построенную на основе РМ-подкодов коразмерности 1}

Докажем полиномиальную сводимость задачи восстановления секретного ключа криптосистемы \(McElSubRM(r,m)\) к задаче восстановления секретного ключа оригинальной криптосистемы Мак-Элиса, построенной на основе кодов Рида--Маллера $RM(r,m)$ (далее~--- \(McElRM(r,m)\)).

Для начала рассмотрим криптосистему с параметрами, удовлетворяющими равенству \(2r=m-1\).
В этом случае \(RM(r,m)=RM(m-r-1,m),\) т.е.
\(RM^{\perp}(r,m)=RM(r,m).\) Тогда \(C^{\perp}_{\alpha, a}(r,m) = RM(r,m) + \{f\},\) для некоторого вектора \(f.\) Но \(RM(r,m)\) раскладывается в объединение смежных классов:
\[
	RM(r,m)=C_{\alpha, a} \cup \{x^{\alpha} \oplus C_{\alpha,a}\}.
\]
Значит код \(C^{\perp}_{\alpha, a}(r,m)\) можно представить в виде объединения четырёх смежных классов:
\[
	C^{\perp}_{\alpha, a}(r,m)=C_{\alpha, a} \cup \{x^{\alpha} \oplus C_{\alpha,a}\}\cup \{f \oplus C_{\alpha,a}\}\cup \{f\oplus x^{\alpha} \oplus C_{\alpha,a}\}.
\]
Мощность каждого класса одинаковая и равна мощности кода \(C_{\alpha,a}(r,m),\) поэтому при случайном выборе вектора \(c\) из кода \(C^{\perp}_{\alpha,a}(r,m)\) c вероятностью \(1/4\) этот вектор будет из смежного класса \(\{x^{\alpha} \oplus C_{\alpha,a}(r,m)\}\).
Тогда пополняя код \(C_{\alpha, a}(r,m)\) этим вектором, получим код \(RM(r,m):\)
\[
	RM(r,m) = C_{\alpha, a}(r,m) + \{c\}.
\]
Следовательно, для восстановления кода \(RM(r,m)\) по исходному коду \(C_{\alpha,a }(r,m)\) потребуется в среднем 4 попытки выбора вектора \(c\).
Таким образом, доказана следующая теорема

\begin{theorem}
	\label{th_2r_equal_m_minus_1}
	Если \(2r=m-1\), то существует полиномиальный вероятностный алгоритм, который по входу задачи взлома криптосистемы \(McElSubRM(r,m)\) строит вход для задачи взлома оригинальной криптосистемы \(McElRM(r,m)\).

\end{theorem}

Пусть теперь \(2r > m-1\).
Рассмотрим код \(C_{\alpha, a}(r,m)\).
Тогда дуальным к этому коду будет код \(C^{\perp}_{\alpha,a}(r,m)=RM(m-r-1,m) + \{f\}\), представленный в виде прямой суммы кода \(RM(m-r-1,m)\) и кода, порождённого ненулевым вектором \(f\), который не ортогонален моному \(x^{\alpha}\), но ортогонален всем векторам кода $C_{\alpha,a}(r,m)$.
Раз скалярное произведение \(\langle f, x^{\alpha}\rangle =1,\) то это означает, что степень булевой функции \(f\cdot x^{\alpha}\) равна \(m\), поэтому если \(\wt(\alpha)\leqslant m - r -1,\) то \(\deg(f) \geqslant r + 1,\) т.е.
\(f\not\in C_{\alpha, a}(r,m)\).
Далее, ограничение на параметры кода позволяет сделать вывод о вложении \(RM(m-r-1,m)\subseteq RM(r,m)\).
Таким образом, код \(C^{\perp}_{\alpha, a}(r,m)\cap C_{\alpha, a}(r,m)\) является подкодом ранка 1 кода \(RM(m-r-1, m)\), который не содержит моном \(x^{\alpha}\).
Значит для некоторого \(a'\) справедливо равенство
\[
	C^{\perp}_{\alpha, a}(r,m)\cap C_{\alpha, a}(r,m) = C_{\alpha, a'}(m-r-1, m).
\]
Причём если \(r'=m-r-1\) и \(2r > m - 1\), то \(2r' = 2 (m-1) - 2r < m - 1.\)
Сформулируем результат в виде теоремы.

\begin{theorem}
	\label{th_2r_greate_m_minus_1}
	Если \(2r>m-1\), то существует полиномиальный вероятностный алгоритм, который по входу \(C_{\alpha, a},\) \(\wt(\alpha)\leqslant m - r - 1\), задачи взлома криптосистемы \(McElSubRM(r,m)\) строит вход для задачи взлома криптосистемы \(McElSubRM(m-r-1,m)\).

\end{theorem}

Пусть \(0< 2r < m-1\).
Тогда на основании теоремы~\ref{SquareOfAlpha2} квадрат кода \(C_{\alpha, a}(r,m)\) либо \(RM(2r,m)\), либо можно считать, что \(\alpha=0\) или \(\alpha=1\) и \(a=0\).
Когда \(\alpha=0\) в коде нет вектора из единиц, поэтому для построения кода \(RM(r,m)\) достаточно пополнить код \(C_{\alpha,a}\) вектором 1.

В том случае, если \(\alpha=1\), в порождающей матрице кода \(C_{1, 0}(r,m)\) будут два одинаковых столбца.
Значит для любых \(r\) и \(m\) код \((C_{1, 0})^{\perp}(r,m)\) может быть представлен в виде прямой суммы \(RM(m-r-1, m)+\{f\},\) где \(\wt(f)=2\).
Рассмотрим \(t=\lceil \frac{m-1}{2r}\rceil\).
Для \(t\) справедливы очевидные неравенства
\[
	\frac{m-1}{2r} \leqslant t \leqslant \frac{m-1}{2r} + 1.
\]
Из которых следует, что \(2(m-1-t\cdot r)\leqslant m-1\) и \(t\cdot r < m-1\) при условии \(r < \frac{m-1}{2}.\) Построим по коду \(C_{1,0}(r,m)\) код
\[
	(t\cdot C_{1,0}(r,m))^{\perp}=(C_{1,0}(t\cdot r, m))^{\perp}=RM(m-t\cdot r-1, m)+\{f\},
\]
\(\wt(f)=2.\)
Тогда согласно утверждению~\ref{th_mceliece} вес любого кодового слова кода \(RM(m-tr-1,m)\) делится на
\[
	2^{\lceil \frac{m-1}{m-tr-1}\rceil} \geqslant 2^2=4.
\]
При этом вес вектора \(f\oplus g,\) \(g\in RM(m-tr-1,m),\) выражается по формуле
\[
	\wt(f\oplus g) = \wt(g) + \wt(f) - 2\wt(f\cdot g) = \wt(g) + 2 - 2\wt(f\cdot g).
\]
Учитывая, что \(0\leqslant \wt(f\cdot g)\leqslant 2,\) получим следующие значения весов вектора \(f\oplus g\)
\[
	\wt(f\oplus g) =
	\begin{cases}
		\wt(g) + 2, & \text{если} \wt(f\cdot g) = 0 \\
		\wt(g),     & \text{если} \wt(f\cdot g) = 1 \\
		\wt(g) - 2, & \text{если} \wt(f\cdot g) = 2 \\
	\end{cases}
\]
Значит, если вес вектора \(c\in (C_{1,0}(tr,m))^{\perp}\) делится на 2 и не делится на 4, то он точно лежит в смежном классе \(\{f+RM(m-tr-1,m)\}.\) А значит вектор \(c\oplus f\) принадлежит коду \(RM(m-tr-1,m).\) Таким образом можно построить не более \(\dim(RM(m-tr-1,m)) - 1\) базисных векторов, т.к.
кодовое расстояние кода, дуального к \(RM(m-tr-1,m),\) равно \(2^{m-tr}>2^{1}\) и поэтому в порождающей матрице кода \(RM(m-tr-1,m)\) не может быть двух одинаковых столбцов.
При этом легко построить базис этого кода, который содержит \(\dim(RM(m-tr-1,m)) - 1\) векторов, согласованных в двух любых координатах.
Последний базисный вектор строится так.
Ищется в коде вектор \(c\), линейно независимый с уже построенными векторами, вес которого делится на 4.
И далее последовательно добавляется к базису вектор \(c\) и вектор \(c\oplus f\).
В одном из этих случаев будет построен базис кода \(RM(m-tr-1,m).\) Теперь, используя следствие~\ref{th_rm_c_alpha}, получить код \(RM(r,m)\) несложно:
\[
	RM(r,m)=\left(RM(m-tr-1,m)\circ (t-1)C_{1,0}(r,m)\right)^{\perp}.
\]

И, наконец, пусть \(C^2_{\alpha,a}=RM(2r,m)\).
В этом случае \((C^2_{\alpha,a})^{\perp}=RM(m-2r-1, m)\) и, значит,
\[
	(C^2_{\alpha,a})^{\perp} \circ C_{\alpha, a} = RM(m-2r-1,m)\circ C_{\alpha, a} = RM(m-r-1, m).
\]
Отсюда
\begin{equation}
	\label{eq_2_closure}
	\left((C^2_{\alpha,a})^{\perp} \circ C_{\alpha, a}\right)^{\perp} = RM(r,m).

\end{equation}
Заметим, что в работе~\cite{couvreur2015cryptanalysis}, конструкция в левой части равенства~\ref{eq_2_closure} называется \emph{2-замыканием} и она используется для построения атаки на кодовые криптосистемы, построенные на основе алгебро-геометрических кодов.
Фактически было установлено, что 2-замыкание кода \(C_{\alpha, a}(r,m)\) является кодом \(RM(r,m)\).

Итак, доказана следующая теорема

\begin{theorem}
	\label{th_2r_less_m_minus_1}
	Если \(2r<m-1\), то существует полиномиальный вероятностный алгоритм, который по входу задачи взлома криптосистемы \(McElSubRM(r,m)\) строит вход для задачи взлома оригинальной криптосистемы \(McElRM(r,m)\).
\end{theorem}

Из теорем~\ref{th_2r_equal_m_minus_1},~\ref{th_2r_greate_m_minus_1},~\ref{th_2r_less_m_minus_1} следует основной результат раздела.

\begin{theorem}
	\label{th_any_param}
	Если \(2r=m-1\) или \(2r<m-1\), или \(2r>m-1\) и секретный ключ \(C_{\alpha, a}\) удовлетворяет условию \(\wt(\alpha)\leqslant m - r - 1\), то существует полиномиальный вероятностный алгоритм, который по входу задачи взлома криптосистемы \(McElSubRM(r,m)\) строит вход для задачи взлома оригинальной криптосистемы \(McElRM(r,m)\).
\end{theorem}

Таким образом, рассмотренная криптосистема, построенная на основе подкодов кода Рида--Маллера коразмерности 1, во многих случаях эквивалентна по стойкости классической криптосистеме Мак-Элиса на кодах Рида--Маллера.

\section{Заключение}
В работе рассматривались подкоды кодов Рида-Маллера коразмерности 1.
Получена классификация произведений Адамара таких подкодов.
С помощью этой классификации авторам удалось установить, что в большинстве случаев задача восстановления секретного ключа кодовой криптосистемы, построенной на основе таких подкодов эквивалентна задаче восстановления секретного ключа этой же криптосистемы, но построенной на самих кодах Рида--Маллера.

В качестве направлений дальнейших исследований можно выделить следующие.
Во-первых, необходимо установить эквивалентность описанных задач и в случае использования параметров, удовлетворяющих свойствам \(2r>m-1, \wt(\alpha) \geqslant m-r.\) Во-вторых, полезным в криптоанализе криптосистем, построенных на различных модификациях кодов Рида--Маллера, было бы построение классификации произведений Адамара для подкодов большей коразмерности.

Интересной также представляется задача описания всех подкодов кода Рида--Маллера, квадрат Адамара которых является кодом Рида--Маллера, а также таких подкодов, из которых операциями умножения Адамара, взятия ортогонального, пересечения и суммы подкодов можно <<восстановить>> код Рида--Маллера.

\begin{thebibliography}{99}

	\bibitem{mceliece1978public}
	\by R.~J.~McEliece
	\paper A public-key cryptosystem based on algebraic coding theory
	\jour Coding Thv
	\vol 4244
	\yr 1978
	\pages 114--116

	\RBibitem{sidelnikov1994open}
	\by В.~М.~Сидельников
	\paper Открытое шифрование на основе двоичных кодов Рида--Маллера
	\jour Дискрет.
	матем.
	\yr 1994
	\vol 6
	\issue 2
	\pages 3--20
	\transl
	\jour Discrete Math.
	Appl.
	\yr 1994
	\vol 4
	\issue 3
	\pages 191--207

	\bibitem{minder2007cryptanalysis}
	\by L.~Minder, A.~Shokrollahi
	\paper Cryptanalysis of the Sidelnikov Cryptosystem
	\jour Lecture Notes in Computer Science
	\vol 4515
	\yr 2007
	\pages 347--360

	\RBibitem{borodin2014effective}
	\by М.~А.~Бородин, И.~В.~Чижов
	\paper Эффективная атака на криптосистему Мак-Элиса, построенную на основе кодов Рида--Маллера
	\jour Дискрет.
	матем.
	\yr 2014
	\vol 26
	\issue 1
	\pages 10--20
	\transl
	\jour Discrete Math.
	Appl.
	\yr 2014
	\vol 24
	\issue 5
	\pages 273--280

	\bibitem{berger2005how}
	\by T.~P.~Berger, P.~Loidreau
	\paper How to mask the structure of codes for a cryptographic use
	\jour Designs, Codes and Cryptography
	\vol 35
	\issue 1
	\yr 2005
	\pages 63-79

	\RBibitem{sidelnikov1992about}
	\by В.~М.~Сидельников, С.~О.~Шестаков
	\paper О~системе шифрования, построенной на основе обобщенных кодов Рида--Соломона
	\jour Дискрет.
	матем.
	\yr 1992
	\vol 4
	\issue 3
	\pages 57--63
	\transl
	\jour Discrete Math.
	Appl.
	\yr 1992
	\vol 2
	\issue 4
	\pages 439--444

	\bibitem{wieschebrink2006attack}
	\by C.~Wieschebrink
	\paper An Attack on a Modified Niederreiter Encryption Scheme
	\jour Lecture Notes of Computer Science
	\vol 3958
	\yr 2006
	\pages 14-26

	\bibitem{wieschebrink2010cryptanalysis}
	\by C.~Wieschebrink
	\paper Cryptanalysis of the Niederreiter Public Key Scheme Based on GRS Subcodes
	\paperinfo PQCRYPTO-2009
	\jour Lecture Notes of Computer Science
	\vol 6061
	\yr 2010
	\pages 61--72

	\bibitem{couvreur2015cryptanalysis}
	\by A.~Couvreur, I.~Marquez-Corbella, R.~Pellikaan
	\paper Cryptanalysis of public-key cryptosystems that use subcodes of algebraic geometry codes
	\jour Coding Theory and Applications
	\yr 2015
	\pages 133-140.

	\bibitem{couvreur2014distinguisher}
	\by A.~Couvreur, P.~Gaborit, V.~Gauthier-Uma{\~{n}}a, A.~Otmani, J.-P.~Tillich
	\paper Distinguisher-based attacks on public-key cryptosystems using Reed--Solomon codes
	\jour Designs, Codes and Cryptography
	\yr 2014
	\vol 73
	\number 2
	\pages 641--666

	\bibitem{otmani2015square}
	\by A.~Otmani, H.~T.~Kalachi
	\paper Square Code Attack on a Modified Sidelnikov Cryptosystem
	\jour Codes, Cryptology, and Information Security
	\yr 2015
	\pages 173--183

	\bibitem{couvreur2015polynomial}
	\paper A polynomial-time attack on the BBCRS scheme
	\by A.~Couvreur, A.~Otmani, J.-P.~Tillich, V.~Gauthier--Umana
	\paperinfo IACR International Workshop on Public Key Cryptography
	\pages 175--193
	\yr 2015

	\bibitem{couvreur2017polynomial}
	\by A.~Couvreur, A.~Otmani, J.-P.~Tillich
	\jour IEEE Transactions on Information Theory
	\paper Polynomial Time Attack on Wild McEliece Over Quadratic Extensions
	\yr 2017
	\vol 63,
	\number 1
	\pages 404--427

	\RBibitem{mcwilliams1979theory}
	\by Ф.~Дж.~Мак-Вильямс, Н.~Дж.~А.~Слоэн
	\book Теория кодов, исправляющих ошибки
	\publaddr Москва
	\publ Связь
	\yr 1979

\end{thebibliography}